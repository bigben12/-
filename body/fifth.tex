% !Mode:: "TeX:UTF-8"

\chapter{总结与展望}

\section{总结}

社会快速发展,科技日新月异,人们的生活每天都有新的变化,
不知不觉中人工智能产品已经进入人们的生活,成为生活中不可缺少的一部分。
人工智能在改善人们生活的同时,也给未来产业的发展提供了无限的开拓空间。
无论是政府,还是各大互联网企业,都十分重视人工智能技术的研究与发展,
在不久的将来,人工智能领域会和人们的生活联系愈加紧密。

计算机视觉作为人工智能研究的一个重要方法,一直以来都受到众多研究人员的关注。
而图像分割是计算机视觉重要的基础研究,对于后续的场景理解任务显得尤为重要。
虽然已经被研究很多年,但仍然是计算机视觉中一个重要课题。

基于像素图像分割处理方法取得了不错的成果。
但是随着拍照设备的不断升级,人们的需求不断增加,需要处理的图像的分辨率不断增大,越来越清晰。
基于像素的传统图像分割方法处理分辨率高的图像,将花费更多的时间。
超像素作为一种图像预处理技术极大减少了处理过程中的计算量和复杂度,
而且更利于局部特征的提取与表达,更有利于帮助定位区域的边界。

本文提出了两种基于超像素的图像分割算法,主要工作体现如下:

 1)提出了一种可以生成超像素和图像分割的端到端可训练网络。
 使用全卷积网络提取图像特征,然后使用可微分聚类算法模块生成精确的超像素。通过使用超像素池化操作获得超像素特征,并且计算两个相邻的超像素的相似度以确定是否合并以获得感测区域。该算法在BSDS500数据集上进行训练与测试。然后从超像素分割结果和图像分割,与最先进的已有算法进行对比实验,证明本文提出算法的高效性。

 2)提出了一种基于Boruvka算法和快速模糊C均值聚类的图像分割。
 在这项工作中,我们使用一种基于Boruvka算法来产生超像素图像。
 由于Boruvka算法可并行化的特性,可以快速高效的产生超像素。此外,在Boruvka算法计算连通分量过程中整合了局部信息,比SLIC和LSC等只利用每像素特征来确定聚类隶属关系的方法得到的超像素更加精确。
 在Boruvka算法获得的超像素图像的基础上,通过计算超像素图像的颜色直方图来实现快速模糊C均值聚类。
 由于超像素图像中不同颜色的数目远小于原始彩色图像,因此计算超像素图像的直方图非常容易。最后,以直方图作为目标函数的参数,实现彩色图像的快速分割。

\section{展望}

本文的主要研究内容是基于超像素的图像分割,图像分割研究是计算机视觉中一个重要的课题。
本文所提出的两种图像分割算法的性能和效率等方面相比于其他算法有了一定的提升,但仍有不足。
今后的研究工作以及需要进一步解决的问题主要包含:

1)第三章提出的基于深度学习的超像素分割和图像分割虽然可以将超像素加入到深度学习中,
但在学习超像素相似性和超像素融合过程中,方式过于简单,仅考虑局部相似性。
所以今后考虑采用更加有效超像素融合方法,采用全局视角的规范化切割,应该会产生更一致的结果。
此外,我们还计划探索算法在其他任务中的应用。

2)第四章提出的基于Boruvka算法和快速模糊C均值聚类的图像分割算法,利用无监督的方法实现图像分割,
省去了训练过程,但其性能方面仍存在很多不足。
此外,在使用过程中需要预先初始化聚类中心数量,且初始参数对最终结果有较大的影响。
所以在今后的研究可以考虑通过算法,自动学习计算聚类中心数量,减少初始化参数对算法结果的影响。

%在将来的工作中,我们计划尝试其他合并算法以获取分割结果。 本文中使用的合并过程相对简单,仅考虑局部相似性。 其他一些程序,例如采用全局视角的规范化切割,应该会产生更一致的结果。 此外,我们还计划探索算法在其他任务中的应用。

