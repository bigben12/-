% !Mode:: "TeX:UTF-8"


\ctitle{基于深度学习的超像素分割和图像分割}  %封面用论文标题,自己可手动断行
\etitle{A Study on Indoor Localization Based on fingerprinting and inertial measurements}
%\etitle{A Study on Subspace Clustering Algorithm \\for High-dimensional Data}
\caffil{天津大学智能与计算学部} %学院名称
\csubjecttitle{工程领域}
\csubject{软件工程}   %专业
\cauthortitle{作者姓名}     % 学位
\cauthor{***}   %学生姓名
\csupervisortitle{指导教师}
\csupervisor{***~~教授} %导师姓名
\ccorsupervisortitle{企业导师}
\ccorsupervisor{***}

\declaretitle{独创性声明}
\declarecontent{
本人声明所呈交的学位论文是本人在导师指导下进行的研究工作和取得的研究成果,除了文中特别加以标注和致谢之处外,论文中不包含其他人已经发表或撰写过的研究成果,也不包含为获得 {\underline{\kai\sihao\textbf{~~天津大学~~}}} 或其他教育机构的学位或证书而使用过的材料。与我一同工作的同志对本研究所做的任何贡献均已在论文中作了明确的说明并表示了谢意。
}
\authorizationtitle{学位论文版权使用授权书}
\authorizationcontent{
本学位论文作者完全了解{\underline{\kai\sihao\textbf{~~天津大学~~}}}有关保留、使用学位论文的规定。特授权{\underline{\kai\sihao\textbf{~~天津大学~~}}} 可以将学位论文的全部或部分内容编入有关数据库进行检索,并采用影印、缩印或扫描等复制手段保存、汇编以供查阅和借阅。同意学校向国家有关部门或机构送交论文的复印件和磁盘。
}
\authorizationadd{(保密的学位论文在解密后适用本授权说明)}
\authorsigncap{学位论文作者签名:}
\supervisorsigncap{导师签名:}
\signdatecap{签字日期:}


\cdate{\CJKdigits{\the\year} 年\CJKnumber{\the\month} 月 \CJKnumber{\the\day} 日}
% 如需改成二零一二年四月二十五日的格式,可以直接输入,即如下所示
\cdate{二零一六年十月}
% \cdate{\the\year 年\the\month 月 \the\day 日} % 此日期显示格式为阿拉伯数字 如2012年4月25日
\cabstract{

图像分割和超像素生成已经被研究很多年,但仍然是计算机视觉中一个重要的课题。
虽然很多先进的计算机视觉的算法已经被用于图像分割和超像素生成方面,
但是没有端到端可训练的算法来实现同时产生超像素和图像分割。
对此,我们提出了一个端到端的可训练网络,它可以同时产生超像素和进行图像分割。
我们的网络架构如图3-1所示。我们使用完全卷积网络和迭代可微聚类算法来获得超像素。
接下来,我们采用超像素池层来获得超像素特征,并以此计算相邻超像素之间的相似度。
如果相似度大于预先设定的阈值,则通过简单的步骤将其合并,得到目标片段。
我们使用BSDS500数据集训练网络,并将我们的结果与最先进的结果进行比较,并得到了很好的结果。

本文提出的网络具有以下特点:
\begin{itemize}
\item 我们的网络是端到端可训练的,可以很容易地组装成其他深层网络结构,以备后续应用。
\item 我们的网络可以产生超像素并得到分割结果,这是更有效的。与现有算法相比,该算法具有优良的性能和更高的精度。
\end{itemize}
}

\ckeywords{图像分割,超像素,深度学习}

\eabstract{

Image segmentation and superpixel generation have been studied for many years, 
and they are still active research topics in computer vision. 
Although many advanced computer vision algorithms have been used for image segmentation and superpixel generation, 
there is no end-to-end trainable algorithm that generates superpixels and segment images simultaneously. 
Specifically, we propose an end-to-end trainable network 
which can generate superpixels and perform image segmentation simultaneously.
The architecture of our network is shown in Fig. 3-1. 
We use a fully convolutional network and the iterative differentiable clustering algorithm to obtain superpixels. 
Next, we adopt the superpixel pooling layer to get the superpixel features, 
with which the similarity between adjacent superpixels can be calculated. 
If the similarity is greater than the preset threshold, 
we merge them according to a simple procedure to get object segments. 
We train the network using BSDS500 segmentation benchmark dataset, 
compare our result with state-of-the-art ones and get good results.

The proposed network has the following characteristics:
\begin{itemize}
\item Our network is end-to-end trainable and can be easily as sembled into other deep network structures for subsequent applications.
\item Our network can generate superpixels and get segmentation results, which is more efficient. The proposed algorithm has excellent performance and has higher precision than existing algorithms.
\end{itemize}

}

\ekeywords{Image segmentation, Superpixel, Deep learning}

\makecover
\clearpage
